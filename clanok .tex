% Metódy inžinierskej práce

\documentclass[10pt,slovak,a4paper]{article}

\usepackage[slovak]{babel}
%\usepackage[T1]{fontenc}
\usepackage[IL2]{fontenc} % lepšia sadzba písmena Ľ než v T1
\usepackage[utf8]{inputenc}
\usepackage{graphicx}
\usepackage{url} % príkaz \url na formátovanie URL
\usepackage{hyperref} % odkazy v texte budú aktívne (pri niektorých triedach dokumentov spôsobuje posun textu)

\usepackage{cite}
%\usepackage{times}

\pagestyle{headings}

\title{Modelovanie bankomatového softvéru v UML\thanks{Semestrálny projekt v predmete Metódy inžinierskej práce, ak. rok 2021/22, vedenie: Ing. Vladimír Mlynarovič, PhD}}

\author{Beáta Belková\\[2pt]
	{\small Slovenská technická univerzita v Bratislave}\\
	{\small Fakulta informatiky a informačných technológií}\\
	{\small \texttt{xbelkovab@stuba.sk}}
	}

\date{\small 6. oktober 2021}



\begin{document}

\maketitle

\begin{abstract}
\ Článok sa zaoberá modelom bankomatového softvéru.  Na začiatku stručne definuje čo je Unified Modeling Language, ktorý je na modelovanie použitý.  Následne pomocou vybraných diagramov demonštruje model softvéru pre bankomat.  Použité diagramy sú sekvenčný diagram, diagram tried a diagram prípadov použitia. 

\end{abstract}




\section{Úvod}

Bankomaty prešli za posledných 60 rokov veľkými zmenami. Od jednoduchý zariadení, ktoré vymieňali jednorazové poukážky za obálky s bankovkami sa stali zariadenia, ktoré požívane denne. Softvér bankomatu sa počas rokov stával čoraz zložitejším, a preto si jeho tvorba vyžaduje sofistikovaný model. 
\newline
Základné informácie o grafickom jazyku, ktorý je použitý na modelovanie ako aj popis jeho diagramov sú v časti ~\ref{uml}.Príklady vybraných UML diagramov pre bankomatový softvér a ich vysvetlenie článok ponúka v časti ~\ref{main}.  Záverečné poznatky a zhrnutie celého článku prináša časť ~\ref{zaver}.


\section{Unified Modeling Language (UML)} \label{uml}
(definícia UML je sformulovaná na základe článku \cite{UML-Uvod})\newline
UML je grafický jazyk, ktorý je štandardom v oblasti analýzy a návrhu pri vývoji softvéru. Vo fáze analýzy pomáha odhadnúť cenu systému a taktiež uľahčuje komunikáciu s klientom, keďže diagramy sú jednoduché na pochopenie. Vďaka UML je jednoduchšie reagovať na zmeny v zadaní klienta. Vo fáze návrhu pomáha riešiť otázku ako bude softvér naprogramovaný.  

\newpage
\Large 
\textbf{Typy diagramov}
\normalsize
\newline
UML v súčasnosti pozostáva zo 14 diagramov. Tie sú rozdelené do dvoch základných kategórií:  diagramy štruktúry (Structure diagrams) a diagramy správania (Behaviour Diagram).
\newline

\hfil\
\begin{tabular}{|l|}
\hline
\textbf{Diagramy štruktúry} \\
\hline
Diagram tried\\
\hline
Schéma komponentov\\
\hline
Objektový diagram\\
\hline
Zložený štruktúrny diagram\\
\hline
Schéma komponentov\\
\hline
Schéma nasadenia\\
\hline
Profilový diagram\\
\hline
\end{tabular}
\hfil  \
\begin{tabular}{|l|}
\hline
\textbf{Diagramy správania} \\
\hline
Diagram aktivít\\
\hline
Diagram prípadov použitia\\
\hline
Stavový diagram\\
\hline
Komunikačný diagram\\
\hline
Sekvenčný diagram\\
\hline
Graf prehľadu interakcií\\
\hline
Synchronizačný diagram\\
\hline
\end{tabular}
\hfil
\newline
\newline
V článku bude pre model bankomatového softvéru použitý sekvenčný diagram (zobrazuje interakcie medzi zákazníkom, bankomatom a bankou), diagram tried (obsahuje všetky triedy, ktoré bude softvér používať a vzťahy medzi nimi), a diagram prípadov použitia. 

\section{Model bankomatového softvér} \label{main}

\subsection{Diagram pípadov použitia } \label{use}

Diagram prípadov použitia je typ diagramu, ktorý zobrazuje vzťah medzi aktérmi(v tomto prípade bankou a klientom) a systémom. Taktiež znázorňuje funkcie, ktoré by mal systému vykonávať. Na obrázku ~\ref{dpp} sú prípady použitia znázornené pomocou elipsy, v ktorej sa nachádza popis práce softvéru. Čiary spájajúce aktérov a prípady použitia predstavujú interakcie medzi nimi.      
\newline
\begin{figure}[h]
\centering
\includegraphics[scale=0.3]{use.png}
\caption{Diagram prípadov použitia\cite{Sample}}
\label{dpp}
\end{figure}


\newpage

\subsection{Diagram tried} \label{class}
Diagram tried je najpoužívanejší UML diagram. Zobrazuje triedy, ktoré bude softvér obsahovať a vzťahy medzi nimi. Diagram musí byť kompletný aby sa podľa neho dal napísať kód a preto musia triedy obsahovať všetky atribúty a metódy. Pod pojmom atribúty si môžeme predstaviť dáta s ktorými bude trieda pracovať. Metódy sú operácie, ktoré bude daná trieda vykonávať.

\begin{figure}[h]
\centering
\includegraphics[scale=0.3]{class.png}
\caption{Diagram tried\cite{Sample}}
\label{dt}
\end{figure}

Na obrázku~\ref{dt}  je ako prvá zobrazená trieda „BANK“. Atribúty,  ktoré bude obsahovať sú kód na jej identifikáciu a adresa.  Metódy,  ktoré banka používa na spoluprácu s bankomatom sú spravovanie a údržba. Znamienko plus pred názvami atribútov a metód znamená, že sú verejné. 
\newline
Ďalšou triedou v diagrame je „ATM “, čiže bankomat. Metódy tejto triedy deklarujú transakcie, ktoré vykonáva.  
\newline
Ako uvádza článok \cite{CLASS}, každá banka potrebuje zákazníkov a preto v diagrame vytvárame triedu „CUSTOMER“. Tým treba priradiť meno a adresu. 
\newline
Týmto spôsobom popíšem aj zvyšok diagramu.

\subsection{Sekvenčný diagram} \label{seq}
V tejto podkapitole článok bližšie popíše obrátok ~\ref{sd}.
\begin{figure}[h]
\centering
\includegraphics[scale=0.3]{seq.png}
\caption{Sekvenčný diagram\cite{Sample}}
\label{sd}
\end{figure}

\newpage
\section{Záver} \label{zaver} % prípadne iný variant názvu



%\acknowledgement{Ak niekomu chcete poďakovať\ldots}


% týmto sa generuje zoznam literatúry z obsahu súboru literatura.bib podľa toho, na čo sa v článku odkazujete
\bibliography{literatura.bib.txt}
\bibliographystyle{plain} % prípadne alpha, abbrv alebo hociktorý iný
\end{document}
