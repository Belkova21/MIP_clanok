% Metódy inžinierskej práce

\documentclass[10pt,twoside,slovak,a4paper]{article}

\usepackage[slovak]{babel}
%\usepackage[T1]{fontenc}
\usepackage[IL2]{fontenc} % lepšia sadzba písmena Ľ než v T1
\usepackage[utf8]{inputenc}
\usepackage{graphicx}
\usepackage{url} % príkaz \url na formátovanie URL
\usepackage{hyperref} % odkazy v texte budú aktívne (pri niektorých triedach dokumentov spôsobuje posun textu)

\usepackage{cite}
%\usepackage{times}

\pagestyle{headings}

\title{Modelovanie bankomatového softvéru v UML\thanks{Semestrálny projekt v predmete Metódy inžinierskej práce, ak. rok 2021/22, vedenie: Ing. Vladimír Mlynarovič, PhD}}

\author{Beáta Belková\\[2pt]
	{\small Slovenská technická univerzita v Bratislave}\\
	{\small Fakulta informatiky a informačných technológií}\\
	{\small \texttt{xbelkovab@stuba.sk}}
	}

\date{\small 6. oktober 2021}



\begin{document}

\maketitle

\begin{abstract}
\ Článok sa zaoberá modelom bankomatového softvéru.  Na začiatku stručne definuje čo je Unified Modeling Language, ktorý je na modelovanie použitý.  Následne pomocou vybraných digramov demonštruje model softvéru pre bankomat.  Použité diagramy sú sekvenčný diagram, diagram tried a diagram prípadov použitia. 

Kľúčové slová : modelovanie,  UML,  sekvenčný diagram, diagram tried, diagram prípadov použitia
\end{abstract}




\section{Úvod}

Motivujte čitateľa a vysvetlite, o čom píšete. Úvod sa väčšinou nedelí na časti.

Uveďte explicitne štruktúru článku. Tu je nejaký príklad.
Základný problém, ktorý bol naznačený v úvode, je podrobnejšie vysvetlený v časti~\ref{nejaka}.
Dôležité súvislosti sú uvedené v častiach~\ref{dolezita} a~\ref{dolezitejsia}.
Záverečné poznámky prináša časť~\ref{zaver}.

\section{Unified Modeling Language (UML)} \label{uml}

UML je grafický jazyk, ktorý je štandardom v oblasti analýzy a návrhu pri vývoji softvéru. Vo fáze analýzy pomáha odhadnúť cenu systému a taktiež uľahčuje komunikáciu s klientom, keďže diagramy sú jednoduché na pochopenie. Vďaka UML je jednoduchšie reagovať na zmeny v zadaní klienta. Vo fáze návrhu pomáha riešiť otázku ako bude softvér naprogramovaný.

\subsection{Typy diadramov} \label{typy}

UML v súčasnosti pozostáva zo 14 diagramov. Tie sú rozdelené do dvoch základných kategórií:  diagramy štruktúry (Structure diagrams) a diagramy správania (Behaviour Diagram).



\hfil\
\begin{tabular}{|l|}
\hline
\textbf{Diagramy štruktúry} \\
\hline
Diagram tried\\
\hline
Schéma komponentov\\
\hline
Objektový diagram\\
\hline
Zložený štruktúrny diagram\\
\hline
Schéma komponentov\\
\hline
Schéma nasadenia\\
\hline
Profilový diagram\\
\hline
\end{tabular}
\hfil  \
\begin{tabular}{|l|}
\hline
\textbf{Diagramy správania} \\
\hline
Diagram aktivít\\
\hline
Diagram prípadov použitia\\
\hline
Stavový diagram\\
\hline
Komunikačný diagram\\
\hline
Sekvenčný diagram\\
\hline
Graf prehľadu interakcií\\
\hline
Synchronizačný diagram\\
\hline
\end{tabular}
\hfil

\flushleft
V článku bude pre model bankomatového softvéru použitý sekvenčný diagram (zobrazuje interakcie medzi zákazníkom, bankomatom a bankou), diagram tried (obsahuje všetky triedy, ktoré bude softvér používať a vzťahy medzi nimi), a diagram prípadov použitia. 

\newpage

\section{Model bankomatového softvér} \label{main}

\subsection{Sekvenčný diagram} \label{seq}

\includegraphics[scale=0.5]{seq.png}

\subsection{Diagram tried} \label{class}

\includegraphics[scale=0.5]{class.png}

\subsection{Diagram pípadov použitia } \label{use}

\includegraphics[scale=0.5]{use.png}

\newpage


\section{Iná časť} \label{ina}

Základným problémom je teda\ldots{} Najprv sa pozrieme na nejaké vysvetlenie (časť~\ref{ina:nejake}), a potom na ešte nejaké (časť~\ref{ina:nejake}).\footnote{Niekedy môžete potrebovať aj poznámku pod čiarou.}

Môže sa zdať, že problém vlastne nejestvuje\cite{Coplien:MPD}, ale bolo dokázané, že to tak nie je~\cite{Czarnecki:Staged, Czarnecki:Progress}. Napriek tomu, aj dnes na webe narazíme na všelijaké pochybné názory\cite{PLP-Framework}. Dôležité veci možno \emph{zdôrazniť kurzívou}.






\section{Záver} \label{zaver} % prípadne iný variant názvu



%\acknowledgement{Ak niekomu chcete poďakovať\ldots}


% týmto sa generuje zoznam literatúry z obsahu súboru literatura.bib podľa toho, na čo sa v článku odkazujete
\bibliography{literatura.bib.txt}
\bibliographystyle{plain} % prípadne alpha, abbrv alebo hociktorý iný
\end{document}
