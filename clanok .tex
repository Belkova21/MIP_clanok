% Metódy inžinierskej práce

\documentclass[10pt,slovak,a4paper]{article}

\usepackage[slovak]{babel}
%\usepackage[T1]{fontenc}
\usepackage[IL2]{fontenc} % lepšia sadzba písmena Ľ než v T1
\usepackage[utf8]{inputenc}
\usepackage{graphicx}
\usepackage{url} % príkaz \url na formátovanie URL
\usepackage{hyperref} % odkazy v texte budú aktívne (pri niektorých triedach dokumentov spôsobuje posun textu)

\usepackage{cite}
%\usepackage{times}

\pagestyle{headings}

\title{Modelovanie bankomatového softvéru v UML\thanks{Semestrálny projekt v predmete Metódy inžinierskej práce, ak. rok 2021/22, vedenie: Ing. Vladimír Mlynarovič, PhD}}

\author{Beáta Belková\\[2pt]
	{\small Slovenská technická univerzita v Bratislave}\\
	{\small Fakulta informatiky a informačných technológií}\\
	{\small \texttt{xbelkovab@stuba.sk}}
	}

\date{\small 6. oktober 2021}



\begin{document}

\maketitle

\begin{abstract}
\ Článok sa zaoberá modelom bankomatového softvéru.  Na začiatku stručne definuje čo je Unified Modeling Language, ktorý je na modelovanie použitý.  Následne pomocou vybraných diagramov demonštruje model softvéru pre bankomat.  Použité diagramy sú sekvenčný diagram, diagram tried a diagram prípadov použitia. \newline
 Kľúčové slová : bankomatový softvér, model softvéru

\end{abstract}




\section{Úvod}

Bankomaty prešli za posledných 60 rokov veľkými zmenami. Z jednoduchých zariadení, ktoré vymieňali jednorazové poukážky za obálky s bankovkami sa stali zariadenia, ktoré požívane denne. Softvér bankomatu sa počas rokov stával čoraz zložitejším, a preto si jeho tvorba vyžaduje sofistikovaný model. 
\newline
Základné informácie o grafickom jazyku, ktorý je použitý na modelovanie,, ako aj popis jeho diagramov sú v časti ~\ref{uml}.Príklady vybraných UML diagramov pre bankomatový softvér a ich vysvetlenie článok ponúka v časti ~\ref{main}.  Záverečné poznatky a zhrnutie celého článku prináša časť ~\ref{zaver}.


\section{Unified Modeling Language (UML)} \label{uml}

UML je grafický jazyk, ktorý je štandardom v oblasti analýzy a návrhu pri vývoji softvéru.  Vo fáze analýzy pomáha odhadnúť cenu systému a taktiež uľahčuje komunikáciu s klientom, keďže diagramy sú jednoduché na pochopenie. Vďaka UML je jednoduchšie reagovať na zmeny v zadaní klienta. Vo fáze návrhu pomáha riešiť otázku ako bude softvér naprogramovaný.  \cite{UML-Uvod}
\newline
\newpage
\Large 
\textbf{Typy diagramov}
\normalsize
\newline

UML pozostáva zo 14 diagramov. Tie sú rozdelené do dvoch základných kategórií:  diagramy štruktúry (Structure diagrams) a diagramy správania (Behaviour Diagram).
\newline

\hfil\
\begin{tabular}{l}
\textbf{Diagramy štruktúry:} \\
Diagram tried\\
Schéma komponentov\\
Objektový diagram\\
Zložený štruktúrny diagram\\
Schéma balenia\\
Schéma nasadenia\\
Profilový diagram\\
\end{tabular}
\hfil \
\begin{tabular}{l}
\textbf{Diagramy správania:} \\
Diagram aktivít\\
Diagram prípadov použitia\\
Stavový diagram\\
Komunikačný diagram\\
Sekvenčný diagram\\
Diagram interakcií\\
Diagram časovania\\
\end{tabular}
\hfil
\newline
\newline

V tubule \ref{tabulka} sú diagramy zoradené podľa popularity. V pravom stĺpci je zapísané, koľko percent zo všetkých softvérov zapojených do prieskumu použilo pri modelovaní daný diagram. Diagram je široko používaný ak má 60 percent a viac. Naopak ak má diagram 40 percent a menej, nepoužíva sa takmer vôbec.  V článku bude pre model bankomatového softvéru použitý diagram prípadov použitia(zobrazuje čo všetko by mal vytváraný systém obsahovať a vykonávať), diagram tried (obsahuje všetky triedy, ktoré bude softvér obsahovať a vzťahy medzi nimi) a sekvenčný diagram (zobrazuje interakcie medzi zákazníkom, bankomatom a bankou v časovom slede ako sa vykonávajú za sebou). 
\newline

\begin{table}[ht]
\centering
\caption{\label{tabulka}Popularita ULM diagramov\cite{tab}}
\begin{tabular}{|c|c|c|}
\hline
\textbf{Priecka} & \textbf{Diagram} & \textbf{Percentil}  \\
\hline
1 & Diagram tried & 100 \\
\hline
2 & Diagram aktivít & 98\\
\hline
3 & Sekvenčný diagram & 97\\ 
\hline
4 & Diagram prípadov použitia & 96\\
\hline
5 & Stavový diagram & 96\\
\hline
6 & Komunikačný diagram & 82\\
\hline
7 & Schéma komponentov & 80\\
\hline
8 & Schéma nasadenia & 80\\
\hline
9 & Objektový diagram & 71\\
\hline
10 & Schéma balenia & 70 \\
\hline
11 &  Zložený štruktúrny diagram & 52\\
\hline
12 & Diagram časovania & 40 \\
\hline
13 & Diagram interakcií & 39\\
\hline
14 &  Profilový diagram & 11\\
\hline
\end{tabular}
\end{table}
\newpage

\paragraph{Historické súvislosti.}
Približne do 70-tych rokov 20. storočia bol vývoj softvéru považovaný za čisto umelecký počin. Systémy sa však časom stávali zložitejšie, takže modelovanie a vývoj softvéru už nebolo možné realizovať kreatívnym spôsobom. Vývojárska komunita potrebovala stabilný a spoločný dizajnový jazyk, ktorý by sa dal na vývoj a modelovanie použiť. Ten pod názvom Unified Modeling Language  v roku 1997 vydala spoločnosť Object Management Group. UML prinieslo jednotný štandardný modelovací zápis, vďaka ktorému môžu teraz IT profesionáli čítať a šíriť systémové plány dizajnu.  \cite{historia}



\section{Model bankomatového softvér} \label{main}

\subsection{Diagram prípadov použitia } \label{use}

Diagram prípadov použitia je väčšinou prvý diagram, ktorý sa pri návrhu systému vytvára. Je to z toho dôvodu, že diagram nerieši ako bude daný systém fungovať, ale čo bude systém vykonávať, teda čo od neho tvorca prípadne klient očakáva. Je zámerne jednoduchý, aby programátor predčasne neuviazol v problémoch s implementáciou systému. Jeho benefitom je, že zobrazuje systém tak, ako ho vidí používateľ a preto uľahčuje komunikáciu s klientom.  \cite{UML-PP}

\begin{figure}[h]
\centering
\includegraphics[scale=0.4]{use case.png}
\caption{Diagram prípadov použitia\cite{Sample}}
\label{dpp}
\end{figure}


Diagram zobrazuje vzťah medzi aktérmi a systémom a funkcie modelovaného systému. 
Na obrázku ~\ref{dpp} sú prípady použitia znázornené pomocou elipsy,  v ktorej sa nachádza popis práce.  Čiary spájajúce aktérov a prípady použitia predstavujú interakcie medzi nimi.  Komunikácia prebieha zľava doprava (pokiaľ nie je šípkou naznačený iný smer).
\newline 
\newpage
Klient môže :   
\begin{itemize}
\item previezť peniaze
\item vybrať peniaze
\item skontrolovať stav na účte
\item odhlásiť sa (pod odhlásením sa rozumieme ukončenie transakcie a vytiahnutie karty z bankomatu)
\item vložiť peniaze
\item prihlásiť sa (pod prihlásením sa rozumieme vloženie karty do bankomatu a zadanie správneho PIN kódu)
\end{itemize}

Prípady použitia pre prevod a výber peňazí majú na seba väzbou \textit{\textbf{include}} naviazanú funkcionalitu na kontrolu stavu účte. Táto väzba sa používa ak sa niektorá funkcionalita v rôznych častiach modelu opakuj. Taktiež sa môže použiť keď je nejaká funkcionalita natoľko dôležitá, že bude v diagrame osobitne zobrazená namiesto toho aby bola iba považovaná za súčasť nejakého prípadu použitia. \cite{UML-PP}
\newline 

Ďalším príznačným prvkom diagramu ~\ref{dpp}  je vzťah \textit{\textbf{extend}} medzi prípadmi použitia pre odhlásenie sa a vytlačenie potvrdenia. Zatiaľ čo väzba include naznačuje, že prípad použitia, z ktorého vychádza obsahuje funkcionalitu, na ktorú ukazuje, pri väzbe extend to neplatí.  Tento vzťah pôsobí opačne.  To znamená, že prípad použitia pre odhlásenie sa nepotrebuje k svojej činnosti prípad použitia, ktorý je k nemu naviazaný väzbou extend.  To však neplatí pre prípad použitia, z ktorého väzba vychádza. Ten nemôže fungovať bez toho aby bol vykonaný prípad použitia, na ktorý ukazuje.  \cite{EXTEND} 

Z diagramu ~\ref{dpp} sa dá teda určiť, že odhlásenie sa vykoná bez ohľadu na to či sa vytlačí potvrdenie.  Avšak potvrdenie sa vytlačí iba v tom prípade, že sa používateľ odhlási. 
\newline 

Posledný špecifický vzťah v diagrame je vzťah  \textit{\textbf{zovšeobecnenia}}, ktorý reprezentuje hrana so šípkou. Označuje, že prípad použitia dedí funkcionalitu nadriadeného prípadu použitia. Šípka ukazuje na nadradený prvok. Diagram teda demonštruje prípad , kedy sa pri odhlasovaní vyberie prípad použitia buď pre kreditnú alebo debetnú kartu a tie zdedia všetky metódy a funkcie pre odhlásenie.\cite{UML-PP}
\newline 

Druhým aktérom v diagrame~\ref{dpp} je banka.  Je to z toho dôvodu, že pri vykonávaní akéhokoľvek prípadu použitia musí bankomat komunikovať s bankovou inštitúciou, ktorá mu poskytuje všetky potrebné dáta o klientovi.  Po uskutočnení operácie taktiež dáta aktualizuje a odovzdá banke. 

\paragraph{Technológia a ľudia.}
Modelovanie a vývoj softvéru je zložitý proces. Obzvlášť keď sa jedná o bankomatový softvér, ktorý pracuje s osobnými údajmi a peniazmi. S narastajúcimi nárokmi a požiadavkami na softvér sa zvyšuje aj požiadavka na vyšší počet kvalifikovaných vývojárov potrebných na jeho tvorbu. Práca vo veľkom tíme však nie je vždy ideálna. 
Z tohto dôvodu sa začali vyvíjať metodiky, ktoré prácu v tímoch uľahčujú. Jednou z nich je aj Agile Scrum. Vďaka nemu dokáže tím dodať funkčný produkt v krátkych cykloch a zaručiť rýchlu spätnú väzbu. Podstatou Scrumu sú 1-4 týždňové šprinty, počas ktorých menšie tímy pracujú na najprioritnejších úlohách až kým nie je produkt dokončený. 



\subsection{Diagram tried} \label{class}
Diagram tried je najpoužívanejší UML diagram. Zobrazuje triedy, ktoré bude softvér obsahovať a vzťahy medzi nimi. Diagram musí byť kompletný aby sa podľa neho dal napísať kód a preto musia triedy obsahovať všetky atribúty a metódy. Pod pojmom atribúty si môžeme predstaviť dáta, s ktorými bude trieda pracovať. Metódy sú operácie, ktoré bude daná trieda vykonávať.\cite{UML-CD}

\begin{figure}[h]
\centering
\includegraphics[scale=0.4]{class.png}
\caption{Diagram tried\cite{Sample}}
\label{dt}
\end{figure}


Na obrázku~\ref{dt}  je ako prvá zobrazená trieda \textit{\textbf{Bank}}. Atribúty,  ktoré bude obsahovať sú kód na jej identifikáciu a adresa.  Metódy,  ktoré banka používa na spoluprácu s bankomatom sú spravovanie a údržba. Znamienko plus pred názvami atribútov a metód znamená, že sú verejné. 
\newline

Ďalšou triedou v diagrame je \textit{\textbf{ATM}}, čiže bankomat.  Metódy tejto triedy deklarujú transakcie,  ktoré vykonáva.  
\newline

Ako uvádza článok \cite{CLASS}, každá banka potrebuje klientov, a preto diagram obsahuje triedu \textit{\textbf{Customer}}. Údaje, ktoré trieda potrebuje sú meno klienta a adresa. Klient nevykonáva žiadne operácie iba vlastní produkty banky.
\newline

Klient môže vlastniť účet a debetnú kartu. Preto bude diagram obsahovať triedy \textit{\textbf{Account}}  a \textit{\textbf{Debit card}}.  Účet má v atribútoch zapísaného vlastníka a typ účtu. Táto trieda môže vykonať kontrolu stavu na účte. Trieda pre debetnú kartu má v atribútoch zapísané číslo karty a vlastníka karty.  Metóda karty je "prístup", pretože umožňuje prístup k bankomatu a tým aj k účtu vlastníka. 
\newline

Ďalšie triedy, ktoré musí diagram obsahovať popisujú transakcie vykonávané počas používania bankomatu. Trieda\textit{\textbf{ATMtransaction}} má tri atribúty – ID transakcie, dátum a typ transakcie. Jej jedinou metódou je aktualizácia. 
\newline

V prípade diagramu ~\ref{dt}  sa predpokladá, že zákazník môže iba vybrať peniaze alebo peniaze previesť na iný účet. Preto obsahuje triedy \textit{\textbf{WithdrawalTransaction}} a \textit{\textbf{TransferTransaction}}.  Tieto dve triedy sú podtriedami \textit{\textbf{ATMtransaction}}  a preto môžu zdediť jej metódy. Transakcia výberu potrebuje údaje o sume, ktorú chce klient vybrať. Potom metóda výber uskutoční vydanie hotovosti klientovi a zdedená metóda aktualizácie aktualizuje stav na účte. Transakcia prevedenia peňazí na druhý účet má iba dva atribúty.  Sumu, ktorú chce klient poslať a číslo účtu, na ktorý má byť suma pripísaná. Trieda po schválení transakcie využije zdedenú funkciu aktualizácia, ktorá aktualizuje stav na účte.\cite{CLASS}
\newline
\paragraph{Udržateľnosť a etika.}
Aby softvér dokázal napredovať a odolávať neustále narastajúcim nástrahám hackerských útokov, je nevyhnutné softvér pravidelne aktualizovať a udržiavať. Okrem opätovného použitia častí kódu sa pri udržateľnosti hovorí aj o nižšej spotrebe zariadenia či používaní ľahšie recyklovateľných materiálov. 
Vývojári sa často ocitajú v situáciách kedy sa budú musieť rozhodovať na základe etiky. V tom nám môžu pomôcť rôzne etické kódexy ako je napríklad Software Engineering Code of Ethics - IEEE-CS a ACM 



\subsection{Sekvenčný diagram} \label{seq}
Posledný diagram, ktorému sa bude článok venovať je sekvenčný diagram. Na základe diagramu tried už vieme aké triedy bude softvér obsahovať a aj aké vzťahy budú medzi nimi. Sekvenčný diagram teraz popíše chronologickú komunikáciu medzi týmito triedami a objektami v čase. Diagram teda zobrazuje ako, a v akom poradí skupina objektov spolupracuje. V sekvenčnom diagrame nie sú kľúčové len kroky aktivity, ale aj vzájomné vzťahy medzi triedami a objektami, ktoré v ňom figurujú. \cite{UML-SD}
\newline

\begin{figure}[h]
\centering
\includegraphics[scale=0.4]{sekvencny.png}
\caption{Sekvenčný diagram\cite{SD}}
\label{sd}
\end{figure}
\newpage

Diagram ~\ref{sd} simuluje výber hotovosti z bankomatu. V hornej časti diagramu sa nachádzajú symboly objektov (obdĺžniky), ktoré reprezentujú aktérov. V tomto prípade budú v diagrame figurovať Klient, Bankomat a Databáza (súčasť bankovej inštitúcie, ktorá poskytuje bankomatu údaje o klientovi). Vertikálne prerušované čiary predstavujú plynutie času. Začínajú tvarom obdĺžnika alebo postavou herca. \cite{SD}
\newline

Klient najskôr vloží platobnú kartu do bankomatu. Kor znázorňuje prvá interakcia na diagrame. Interakcie je znázornená plnou šípkou s plnou čiarou. Tento symbol sa používa keď odosielateľ musí čakať na odpoveď. 
Následne musí bankomat overiť, či je číslo karty platné, a preto pošle kvôli overeniu správu do databázy. Pri databáze sa vytvorí aktivačná lišta, ktorá ukáže, že databáza je počas tohto obdobia aktívna. Po overení pošle bankomatu správu ,že je karta v poriadku. Táto správa bude odoslaná späť na konci aktivačnej lišty\cite{SD}
\newline

Teraz bankomat požiada používateľa aby zadal svoj PIN kód. Po zadaní kódu musí bankomat opäť skontrolovať databázu a overiť, či je kód správny. Tá následne PIN kód overí a pošle bankomatu správu. Pri overovaní údajov v databáze bude na diagrame opäť znázornená aktivačná lišta. Keď je všetko v poriadku bankomat ponúkne klientovi ponuku možností. 
\newline

V tomto bode prebieha výmena interakcií medzi klientom a bankomatom. Klient si z ponúknutých možností zvolí možnosť výberu hotovosti, bankomat sa spýta na čiastku a klient zadá čiastku, ktorú chce vybrať. Počas celej interakcie je pri klientovi znázornená aktivačná lišta.  
\newline

Ako pri predošlých interakciách aj teraz musí bankomat poslať správu do databázy a overiť, či má klient na účte dostatok prostriedkov. Teraz však diagram zobrazí oba prípady odpovede. Aby bolo z diagramu zreteľne jasné, aké interakcie budú v oboch prípade prebiehať, sú znázornené oddelene. Pre oddelenie viacerých možností odpovede sa v sekvenčnom diagrame používajú slučky (prázdne obdĺžniky, ktoré majú v pravom hornom rohu štítok pre text). V prípade diagramu ~\ref{sd} je v štítku text alt = alternatíva.  \cite{SD}
\newline

Ak má klient na účte dostatok prostriedkov bankomat vydá hotovosť, klient si ju vezme a bankomat mu nakoniec vráti platobnú kartu (v niektorých prípadoch bankomatom vracia kartu pred vydaním hotovosti, takže interakcie v tejto časti budú na diagrame vymenené).
\newline

V opačnom prípade databáza pošle bankomatu správu o tom, že klient nemá na účte dostatok prostriedkov. Bankomat následne zobrazí klientovi správu o zamietnutí transakcie a vráti mu platobnú kartu. 
\newline

\section{Záver} \label{zaver} 

Pri tvorbe zložitých softvérov sa môže stať, že sa počas vývoja zmenia požiadavky klienta alebo sa vplyvom nečakaných okolností softvérový projekt predraží či prestane mať jasný zámer. Aby bol softvérový projekt úspešný, je potrebné pred samotnou realizáciou pripraviť jasný a prepracovaný model. O to viac keď sa jedná o softvér denného použitia, ktorý pracuje s osobnými údajmi a peniazmi. Článok demonštroval model pre bankomatový softvér pomocou troch UML diagramov. Je to dobrý začiatok, ale pre vytvorenie kvalitného modelu to určite nestačí. Dobrými pomocníkmi pri modelovaní softvéru sú aj diagram aktivity a stavový diagram. Súčasťou dobrého plánu sú aj jasne stanovené požiadavky na systém a rozpočet. 

%\acknowledgement{Ak niekomu chcete poďakovať\ldots}


% týmto sa generuje zoznam literatúry z obsahu súboru literatura.bib podľa toho, na čo sa v článku odkazujete
\bibliography{literatura.bib.txt}
\bibliographystyle{plain} % prípadne alpha, abbrv alebo hociktorý iný
\end{document}
